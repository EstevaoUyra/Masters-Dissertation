\chapter{Objectives and Hypothesis}
\label{chap:hypothesis}

    The current work aims to assess how neural activity changes through learning, with respect to the representation of time. Specifically, our objectives are divided into one methodological and one practical.

\begin{itemize}
    \item Methodological: To understand how the choice of classifier may impact results. Are the results highly dependent on the kind of classifier used for analysis?
    \item Practical: To understand how the quality of representation changes when comparing naive and proficient rats.
\end{itemize}

    With respect to the methodology, we expected the choice of classifiers to influence the results as a matter of degree, but not qualitatively. That is: classifiers could each have different performances, but all will be able to find significant patterns.

    As to the practical aspect, we expected the quality of time representation to increase as a function of proficiency, as depicted in figure \ref{fig:hip}, which means that after learning the performance of classifiers should increase. 

\begin{figure}
    \centering
    % \includegraphics[width=\textwidth]{figuras/hipot.PNG}
    \caption[Visual sketch of our hypothesis.]{Visual sketch of our hypothesis. We expect that when the animal is learning the representation is weak or absent, and is stronger when the animal is performing well.}
    \label{fig:hip}
\end{figure}
