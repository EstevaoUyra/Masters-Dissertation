\chapter{Experiment description}
\label{chap:experiment}

\section{The task}
The subjects were adult naive male Wistar rats (12 - 16 weeks old, 380 - 400g). The animals were housed individually in a 12h light/dark cycle with the light turned on at 7am, and put under food deprivation to gradually reach and then maintain 85\% of their free-feeding weight. Animals were handled according to the experimental protocol approved by the Committee of Ethics in Animal
Usage from UFABC (CEUA-UFABC). All procedures were conducted during the light cycle.

Animals were trained in a operant chamber developed in our laboratory, totally controlled by an Arduino uno, and made of acrylic plastic. The box has a \textit{nosepoke}, which is a hole equipped with an infrared emitter-sensor that detects when the animal pokes inside. In the same wall there is a drinker coupled to a lickometer, to which the animal has its access limited by a gate controlled by a stepper motor.

\begin{figure}
    \centering
    \includegraphics[width=\textwidth]{figuras/methods/tarefa_eli.png}
    \caption[Our DRRD task]{Our DRRD task. If the animal stays in the nosepoke for longer than the criteria, it can collect a reward. Taken with permission from \cite{Eliezyer2018}}
    \label{fig:task}
\end{figure}

\clearpage

First, the animals were shaped to nose poke the hole through a continuous schedule of reinforcement (CR). To conclude this phase the animal had to nose poke the hole at least 100 times on a 60 minutes session. After reaching this performance,
on the next day, the animals were trained with a Differential Reinforcement Response Duration (DRRD) procedure: The trials were self initiated and self ended. In order to receive reward (four licks of a 50\% glucose solution), the animal had to poke the hole and hold for a time equal or longer than the defined criterion of 1500ms. If the animal didn't reach the time criteria there is no reward, and it is free to start a new trial at any time. All sessions were video recorded.