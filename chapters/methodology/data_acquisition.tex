\section{Acquisition}
\label{sec:acquisition}

    Arrays of electrodes were chronically implanted by stereotaxic surgery. For the first group of animals, 32-electrodes array made by TDT (Tucker-DavisTechnologies, EUA) were implanted in the right mPFC (AP: 3.2mm, ML: 1mm, DV: 3.3mm) at UFABC's Electrophysiology lab. For the second group, 32 custom manufactured electrodes were divided, and half implanted on the Dorsal Striatum (AP: 0.28 $\sim$ 2.28 mm, ML: -1.8 $\sim$ -3.5 mm ) and half in the Prefrontal Cortex (AP: 2.52 $\sim$ 4.20 mm ML: -0.3 $\sim$ -1.3 mm). All coordinates were assessed via craniotomy, and measured from the bregma.
    
    The signal was recorded using TDT equipment while the animals were performing the CR or the DRRD task, with a 25KHz rate acquisition. The unit signal was filtered between 1 to 5 kHz. Spike sorting was first performed online and then by the \textit{Open sort} software also developed by TDT. Spike activity was analyzed for all cells that presented the peak of firing rate above 0.1 Hz. 

\begin{table}[htp]
    \centering
    \begin{tabular}{l|l|c|c|c|c|c}
        Animal & Group & Trials & Rewarded & \% Rewarded &  mPFC Neurons & STR Neurons \\\hline
        3 & 2 &  543 / 625  &    278 / 410 &   51 / 65 &       4 / 9    &   10 / 13 \\
        4 & 2 &  436 / 381  &    218 / 222 &  50 / 58  &      25 / 24   &   23 / 11 \\
        5 & 2 &  509 / 437  &    156 / 203 &  30 / 46  &      3 / 5     &   12 / 17 \\
        6 & 2 &  936 / 699  &    404 / 382 &  43 / 54  &      2 / 2     &   0 / 3   \\
        7 & 1 &     1192    &    654       &    54     &     23       &     -     \\
        8 & 1 &      801    &    454       &    56     &     28       &     -     \\
        9 & 1 &      932    &    504       &    54     &     14       &     -     \\
        10& 1 &     1671    &    1030      &    61     &     68       &     -     \\
    \end{tabular}
% \toprule
    \caption[Dataset specifications]{Dataset specifications. The animal's number does not represent any ordering. Group 2's neural activity was measured in two regions, and in two sessions, while group 1's only in the mPFC, and in a single session. The slash in animals of group 2 separate the number of neurons recorded in the first session (at left) from the second session (at right).}
    \label{tab:subjects}
\end{table}