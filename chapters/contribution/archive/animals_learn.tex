    Before going into an analysis of the electrophysiological activity, we validated the animals' behavior and learning. First of all, we needed to ensure that animals were engaged in the task, desiring to receive their reward as soon as they were delivered. That way, increases in the waiting time inside the nosepoke could be attributed to learning and not attributable to satiation. The latency between subsequent trials -- the intertrial interval, or ITI --  served as a surrogate for this engagement as shown in figure \ref{fig:iti}. 
    Figure \ref{fig:iti} also shows us that somewhere near the end of the session, the animals increased their ITI, contrasting with ITIs at the beginning of the session that were typically around seconds. While this suggests that sessions could have been shorter, the important fact for our subsequent analysis is that we did assess the animals' engagement, removing later trials with big ITI, and leaving only trials with engagement to our subsequent analysis. This was done via visual inspection, in which we generously removed trials to ensure we wouldn't pollute posterior analysis with these unengaged trials.
