\chapter{Animals Learn fast}
\label{chap:learning}

% Recent findings from our group suggest that animals can learn the DRRD in a single 50-minute session \cite{ReyesDRRD}. We designed our methodology based on that expectation. Hence, we compared the distribution of the response duration in different stages of the session 

We designed our methodology based on the expectation that rats learn the DRRD task in a single session, as previous findings from our group showed~\cite{ReyesDRRD}. To verify whether this assumption holds to our data, we compared the distribution of response duration in different stages within each session. However, to make sure rats were engaged in the task, aiming to receive rewards as frequently as possible, we analyzed the duration of periods between trials (intertrial intervals, ITI). We hypothesized that those intervals indicate the level of engagement of rats while performing the task -- more extended periods between nose pokes indicate lack of interest in reward and less engagement in the task.

As described in the session \ref{sec:acquisition}, we analyzed two distinct data sets, one in which the animals trained for several hours and the other in which they trained for fewer hours but with more than one day of training.  Hence we gathered data differently for these two groups of animals. The average number of trials in the long sessions was $1149 \pm 384$ trials and provided about $660 \pm 260$ correct trials. In turn, in short sessions consisted in $570 \pm 181$ trials and with a smaller amount of correct trials ($284\pm 100$) -- about half the number obtained in long sessions. Thus, for group 1, of animals which performed a single long session (rats 7, 8, 9 and 10), we split each session into its first and second half. In turn, for group 2, of animals which performed two shorter sessions (rats 3, 4, 5 and 6), we analyzed the first and second session separately. This procedure thus leads to proportional sets of trials divided into earlier and later parts for each animal.

%Our methodology is centered on the expectation that learning our task occurs as fast as in a single session. To confirm this expectation we compare the distribution of responses across learning stages. Specifically, we divide the session into the first and second half for the group with long sessions, and into the first and second sessions for the group with short sessions. This is because for the first group we had long sessions of more than 1000 trials $(1149 \pm 384)$ with sufficient correct trials $(660 \pm 260)$. For the second group of animals, in which sessions were half as long (with $570 \pm 181$ trials), there were only and $284\pm 100$ correct trials, less than half compared to the first group. This comparison is preceded, however, by an assessment of the engagement of rats: we want to assure that animals intend to obtain reward as frequently as possible.

\section{Rats disengage in long sessions}
    A central assumption of our study is that animals maximize the amount of reward received, as assumed in conditioning studies in general. In our task, animals have to stay in the nosepoke long enough to receive their reward but holding too long delays the reward. When animals seek reward as fast as possible, we expect that they to remain in the nosepoke as little as possible to receive the reward. In this case, increases in response durations can be attributed to learning of the underlying time interval. Moreover, because the animals can start new trials whenever they want, we expect them to do so with little delay. Alternative reasons for increased response times could be diminished interest in the reward, satiety or tiredness. We hypothesized that, if the animal disengages the task, the interval between trials (ITI) should also increase.

    \begin{figure}[h]
        \centering
        \includegraphics[width=\textwidth]{figures/inter_trial_boxplot.png}
        \caption[Time between trials is in the order of seconds]{Time intervals between trials are in the order of seconds. The distribution of all inter-trial intervals for each rat in boxplot quantiles. The horizontal trace above each box shows the last point up to 1.5 IQRs (Inter Quartile Ranges) of the .75 quartile, a common threshold to define outliers. The percentage right below that line shows how many of the points are below the threshold.}
        \label{fig:iti_box}
    \end{figure}
    
    In figure \ref{fig:iti_box} we can see that most of the intervals are below 20 seconds. Specially in rats 6 and 10, more than 90\% of ITIs were under 10 seconds. Although there is variation, intervals are in the order of low tens of seconds, and thus seem to be consistent with reward seeking. On the other hand, there are many outliers: whereas figure \ref{fig:iti_box} showed up to 30s, there are some intervals with more than 10 minutes between trials. If these large intervals were distributed all along the session, this could complicate our findings. However, if they are concentrated later in the sessions, they may be indicative of natural tiredness or satiety of a previously engaged animal.
    
    \begin{figure}[ht!]
        \centering
        \includegraphics[width=\textwidth]{figures/inter_trial_alongtrial.png}
        \caption[Rats disengage in long sessions]{Rats disengage in long sessions, increasing their resting period between trials, waiting for minutes without engaging in the nosepoke. Each row shows the ITI of a single animal, ordered by trial, and normalized by the session duration. The upper four rows correspond to animals in the group 2, with two shorter sessions, and two respective columns. The red demarcation displays the limit of engagement, measured as the first sum of 5 or more minutes ITI in 5 consecutive trials. Posterior trials have been removed from subsequent analysis,}
        \label{fig:iti}
    \end{figure}
    
    
    To understand better how this long outlier ITIs are distributed along the session, we plotted the ITIs of each trial in sequence, in figure \ref{fig:iti}. The figure shows the two shorter sessions of animals from the group 2, and the single long session from group one. We can see that indeed the ITIs increase later in the session for almost all recorded sessions, including some of the shorter. There is some heterogeneity in the disengagement patterns: In rat 9, many medium ITIs of 1 to 5 minutes can be seen close together, with a single bigger peak. In rat 8 there are less medium ITIs, and some peaks of 10 or more minutes are separated by small ITIs (< 1 min). In rat 5, a single big peak is surrounded by small ITIs.
    
    Disregarding the heterogeneity in ITI increase, longer ITIs are more frequent later in the session, supporting the hypothesis that animals get tired (or satiated) and disengage the task after some time. We defined a threshold for rejecting trials after a ``disengagement point'' in a reproducible manner. For such, we calculated the sum of ITIs for each 5 consecutive trials, and excluded all trials after the first sum above 5 minutes. This could be reached by a single trial with ITI bigger than 5 minutes, by five consecutive trials with 1 min of ITI each, by one ITI of 3 minutes followed by another one of 2 minutes, and so on. Figure \ref{fig:iti} shows the point at which sessions were pruned according to this criterion (red dashed line).
    
\section{Learning progression is diverse}
    \begin{figure}[ht!]
        \centering
        \includegraphics[width=\textwidth]{figures/behavior_group_1_with_avg_bold.png}
        \caption[Behavior across single session]{Behavior across single session, for four subjects. The circles and triangles represent the incorrect (shorter than 1.5s) and correct trials (longer than 1.5s) respectively. An orange line represents the moving average of 50 consecutive trials. The density plot on the right shows the distribution of responses in the first and second halves of the session. Rewarded trials are shaded, in such a way that the increase in rewarded trials is proportional to the difference in blue minus red shadings.}
        \label{fig:behavior}
    \end{figure} 
    
    \begin{figure}[ht!]
        \centering
        \includegraphics[width=\textwidth]{figures/behavior_group_2_with_avg_bold.png}
        \caption[Behavior across two sessions]{Behavior across two shorter sessions, for four subjects. The circles and triangles represent the incorrect (shorter than 1.5s) and correct trials (longer than 1.5s) respectively. An orange line represents the moving average of 50 consecutive trials. The density plot on the right shows the distribution of responses in the first and second sessions. Rewarded trials are shaded, in such a way that the increase in rewarded trials is proportional to the difference in blue minus red shadings.}
        \label{fig:behavior2}
    \end{figure}
    
    We show the responses for each group of animals in figures \ref{fig:behavior} and \ref{fig:behavior2}. The responses are highly variable, from milliseconds to seconds, in such a way that there are rewarded trials even early in the beginning of the session. In some animals the variability in response duration decreases perceptively (e.g. 7 and 10) and in one of them it increases (i.e. 9). This can be best seen in the density plots, showing distribution of responses in the first and second half.
    
    The moving average drifts up and down, in some rats more than in others. There are rats in which the drift is closer to monotonic. In them, variation between near trials seems smaller, for example rats 6, 8 and 9. Specially in rat 8, the moving average changes smoothly, contrasting with rats 4, 7, and 10, whose moving average has peaks early on and descends afterwards. Rat 10 reaches a moving average duration double the criterion in the first half of the session, with huge variance in its responses, before reducing the duration and variance, and centering responses around the criterion of 1.5 seconds.

\section{Rewarded responses show learning in few hundred trials}

    Looking at figures \ref{fig:behavior} and \ref{fig:behavior2}, we can see that the proportion of rewarded trials increases with training. The proportion is just the count of rewarded trials over the total number of trials. Rewarded trials are those with duration bigger than the criterion of 1.5s, and can be seen in the upper part of each behavior plot. The upper part of the density plots, above the dashed line, indicates this proportion in the area under each curve. We tested the significance of this increase separately for each group using paired-sample t-tests, and found that both are significant. This was calculated as follows: For group 1, we compared the ratio of reinforced trials in the first versus second half of the session, evaluating to p = 0.037 (T=3.564). For group 2, the first session was compared with the second, evaluating to p = 0.0002 (T=20.756). In either case, all trials after the point of no-engagement were excluded before calculation of the ratio of reinforcement.
    
    In sum, although the precise progression of response times is different for each animal, they all increase their rate of correct responses. We compared here two distinct learning stages, showing animals indeed learn and change their behavior really fast. The same learning stages will be compared with respect to their neural activity and time representation in chapter \ref{chap:rep_changes}
    