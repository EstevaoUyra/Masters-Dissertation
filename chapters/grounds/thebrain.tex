%!TEX root = ../dissertation.tex
\chapter{The Brain}
\label{cap:thebrain}

The sensory apparatus of single cell organisms is directly linked to their motor apparatus, meaning that structures capable of detecting perturbations are tightly coupled to structures capable of generating movement \cite[p.~149]{maturana1987tree}. In protozoa, for example, the same flagellum that moves it in the environment also detects obstacles, while some bacteria have chemotaxing mechanisms that change direction of movement, increasing or reducing tumbling rate, in direct response to changes in sugar concentration \cite[p.~147-149]{maturana1987tree}. The coupling between sensory and motor surfaces (i.e. what to do in response to perceived environment), in these direct links, is inflexible %\change{this phrase is unclear. What link is this?}.

In contrast to direct sensory-motor connections, the nervous system appears as an intermediate component between the surfaces of interaction of an organism with the surrounding environment. A nervous system endows its owner with structural plasticity, thus enabling learning via changes in this intermediate step between sensory and motor surfaces, producing bigger behavioral repertoires and complex behavior \cite[p.~175]{maturana1987tree}. This comes at the expense of a lot of energy: up to 20\% of human total resting metabolism \cite{attwell2001energy}, which is used mainly by \textit{neurons} \cite{zhu2012quantitative}, a cell type that receives, modulates, and sends signals. The reason why much of our energy budget is directed towards this single type of cell, can only be explained by the importance of its activity for the organism.

The introduction will be divided into two parts. In \ref{sec:theory} we introduce some general ideas about the nervous system's role, directing to the idea of representation and into the domain of time perception. Next, in section \ref{sec:da} we give a brief introduction to the functioning of neurons and the type of data which can be collected from their activity, followed by a methodological introduction to the data analysis techniques using supervised learning, applied in this work.

\section{What the brain really does}
\label{sec:theory}
We know that the nervous system mediates the interaction between a living system's sensory inputs, i.e. what is perceived from the environment, and the system's output, e.g. movement. Based only on excitatory and inhibitory connections between neurons, even in the simplest unidirectional case, it is possible to generate fairly complex behaviors, which can be interpreted as fear responses, aggression, or even logic \cite{braitenberg1986vehicles}. % Possivelmente uma imagem do vehicles, tipo figura 5 pg13

In extremely simple circuits, such as the ones described above, it may be possible to generate mechanistic accounts of this mediation, providing effective explanation to phenomena like reflex arcs. With little complexity added, these explanations can get really intricate, as in the case of the much studied Somatogastric Ganglion (STG) of the lobster. In this system, oscillatory patterns are generated by a circuit of 11 neurons \cite{selverston2009neural}. Despite of this circuit's smallness, it took 30 years from their discovery to the development of mechanistic explanations for its functions \cite{bal1988pyloric, selverston2009neural}. In addition, there are multiple combinations of connections between neurons that give rise to the same rhythms, many of which are found \textit{in natura} \cite{prinz2004similar}. 

% https://www.nature.com/articles/nn1352

This raises an important issue in modern neuroscience, viz. that if we want to increase the predictive power of our models, we have to step away from the implementional level, and delve deeper into theories of brain function and into general properties of brain activity \cite{gerstner2012theory}. One remarkable proponent of this formalism is the Free Energy Principle \cite{friston2009free}: it states how self-organizing systems -- an in special the brain -- can resist the tendency to disorder. According to the Free Energy Principle, to maintain itself in the narrow band of states that are consistent with physiological bounds, an organism must have an internal representation of the environment, constantly updated by new evidence \cite{friston2009free}.
% What we KNOW
% There are many ways through which 

%inclination angle
\section{Representation}
\label{sec:representation}
The notion that representation is central to cognition is an old one \cite[p.~134-140]{rosch1991embodied}, and provides the grounds for some methods of assessing brain function, one of which is the search for neural correlates of external quantities. We know that there are neurons in the primary visual cortex that represent (i.e. encode) inclination angles of bars \cite[p.~13]{dayan2001theoretical}, neurons in the primary motor cortex that represent the angle of reaching movements \cite[p.~14]{dayan2001theoretical}, and so on.

Single neuron representations have been specially studied by the vision community \cite{deyoe1988concurrent,bell1997independent,ito2004representation,lee2008sparse}, but recent contributions have taken this approach much beyond the sensory realm, to the more abstract domains of \textit{space} and \textit{time} \cite{eichenbaum2014time}. \textit{Place cells} are neurons that fire in specific locations of some environment \cite{foster2006reverse}, and are mainly present in the Hippocampus \cite{o1979review}, bestowing the brain with a whole new dimension along which to relate its activity \cite{eichenbaum2014time}. In the same way, \textit{time cells} fire in specific times after some event \cite{tiganj2016sequential, eichenbaum2014time}, and may serve as temporal basis for representations of the world \cite{ludvig2008stimulus}, being complemented by \textit{ramping neurons}, that linearly increase or decrease their activity during a time interval \cite{morrison2009convergence, kim2013neural, tiganj2016sequential, parker2016timing}.

Although there are many examples of quantities encoded in single neurons, such as aforementioned, it is not necessary for representations to have particular neurons with such interpretable roles. Represented quantities may be encoded in less explicit ways, such as in post-synaptic sensitivity \cite{motanis2018short}, through neuromodulation \cite{friston2009free, friston2010free}, or in neural populations \cite{shamir2014emerging, quiroga2009extracting}. The former two are not directly detectable from neural activity, thus although they affect behavior \cite{wolff2017dynamic} we will not account for them in the present work.
% accumulators bueti2011physiological, wittmann2010accumulation

Population codes have the capacity to carry much more information than single neuron codes \cite{hardy2016neurocomputational}, but since population patterns are by definition multivariate, their study requires more ellaborate techniques \cite{quiroga2009extracting}, such as Machine Learning. There are many ways in which a given neural population may encode information \cite{quiroga2009extracting, shamir2014emerging, mello2015scalable}, and in the specific case of encoding temporal information, several models have been developed to account for characteristics and biases present in empirical data \cite{hardy2016neurocomputational}. The motivation and characteristics behind some of these models will be presented below.

The representations we discuss here are taken in the weak sense: any internal pattern that reliably covaries with an external one may be said to represent, or \textit{encode} it \cite{rosch1991embodied}, and in this sense there is no shortage of time representations in the brain. On the other hand, the representations that interest us are those actually causing impact in subsequent behavior, and for these end they must also be accessible to processes needing them.
% population and trajectory