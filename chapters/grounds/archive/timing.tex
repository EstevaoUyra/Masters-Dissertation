    
    
    
    Perception is also inseparable from time: differently from the way percepts can be visual or auditory, they are always temporal, because time is an intrinsic dimension of  
 
 
    cerebellum, although this is still disputed \cite{ohmae2017cerebellar, nichelli1996perceptual}.
 
    Since we believe there are neural causes for behavior, which certainly predict behavior better than time itself, this internal causes must be ignored by the definition. 
 
    We have not found definitions of timing with respect to internal states, and parsimoniously propose the following one: 
    \begin{quote}
    We call timing-related activity all activity that is necessary for the performance of timing behavior, and not necessary for the most similar task with no timing.
    A timing task is 
    \end{quote}
    
    
    
        To account for timing behavior, especially in the scale of seconds, a variety of models has been devised and discussed by the community. There are too many models to consider an exhaustive listing \cite{}, and it is out of the scope of this work. We have thus chosen to discuss three algorithmic models of increasing complexity: SET, LeST, and RWDDM. SET (from Scalar Expectancy Theory) is the most popular model \cite{}, mainly beause of its simplicity, that makes it great to build intuition and start discussing the necessary components of any model. LeST has been proposed as a descendant of the SET model, adding features of another popular LeT model while retaining much of its simplicity. At last, RWDDM - Rescorla-Wagner Drift Diffusion Model is a very recent proposal, which is more general and has more connections with the implementation level.
    
    \subsection{SET}
        The SET model consists of a pulse generator, an accumulator, a 
        
    \subsection{LeST}
        The LeST model 
    
    \subsection{RWDDM}
    
    
    
    
    
    
            Alternatively to the central mechanism, if neural networks encode time as an intrinsic property, no such core would exist, as local circuits can tackle the learning process via state-dependent changes in network dynamics \cite{mauk2004neural, paton2018neural, buonomano1995temporal, motanis2018short}. 
        
        Somewhere between these two extremes of disperse local mechanisms versus central encoding, core regions may depend on further aspects beyond time scale alone. The notion that the size of an interval is sufficient to delineate the mechanism of state representation that will be used in a given timing task is not uncontested \cite{van20168}. Alternatively, other characteristics, like presence and continuity of movement, may facilitate the use of less expensive, automatic timing mechanisms, in opposition to attention based, cognitively controlled mechanisms \cite{lewis2003distinct}, even though increasing the size of an interval does require more flexible, cognitively controlled systems. Beyond the sensorimotor processing, the modality of the task (e.g. auditory vs visual) and number of intervals also affect temporal variability, changing Weber's constant \cite{merchant2008we}.