\chapter{Learning}
    Learning is the process of changing behavior through experience. One of the oldest subjects in the cognitive sciences \cite{}, learning is related to most 
    
\section{Instrumental learning}
     % alguma citação do Pavlov
    
        
    \subsection{Dual systems guide behavior}
        There are two distinct forms of behavior that can be learned. Goal-directed behavior are those in which the animal knows the consequence of its action (belief), and this consequence is wanted (desire). In contrast, habitual behavior are those in which the animal is insensitive either to belief or desire, acting in a predetermined manner. Take the example of pressing a lever for food pellets: if the animal is not hungry (no more desire) or the lever stops yielding access to food pellets (no more belief), it will stop pressing if the behavior was goal directed, but continue pressing if it was habitual \cite{}. In sum, while goal-directed behavior is flexible to changes in rewards, habitual behavior is resistant to change \cite{}.
        
        As exemplified above, the same behavior may be goal-directed or habitual. Generally, over-training makes a behavior habitual, or more specifically: makes the habitual system the main controller of behavior \cite{}. Whereas the amount of training may change the main controller, this does not mean that the system is independent of the behavior: there are some hard to automate, and some hard to control cognitively \cite{}. Both systems work conjointly to guide behavior, and their relative contribution is dependent on the reinforcement schedule in addition to the complexity of the task \cite{dickinson2015instrumental}.
        
        Beyond the belief-desire account, we can frame the dual systems in terms of associations: While Response-Outcome (RO) associations give rise to Goal-directed behavior, Stimulus-Response (SR) associations give rise to habitual behavior \cite{}. One third option is in terms of reinforcement learning: model-based learning corresponds to goal-directed behavior while model-free learning is habitual. 
        Each system has its own neurobiological substrate: while the goal-directed is dependent on dmPFC and dSTR, the automatic is dependent on vmPFC and vSTR \cite{dickinson2015instrumental}. 

    \subsection{Dual systems have distinct neural substrates}
        Studies on the corticostriatal role in decision making have already proposed the coexistence of two heterogeneous learning processes \cite{balleine1998goal, balleine2007role, smith2013dual}: one of them consists of an automatic/habitual Stimulus-Response learning, strengthened by reinforcers, while the other reflects a goal-directed/flexible Response-Outcome learning \cite{dickinson2015instrumental}. The neurobiological substrates of these two processes have been dissociated, wherein the goal-directed is dependent on dmPFC and dorsal Striatum, and the automatic is dependent on vmPFC and ventral Striatum \cite{dickinson2015instrumental}.

        Studies on the vmPFC's importance to learning, in the field of fear extinction, point out to its role in retaining learning \cite{phelps2004extinction}, while it seems necessary to performance in gambling tasks \cite{rogalsky2012risky}. Distinctly, dmPFC's role has been verified to bolster learning but not performance \cite{balleine2007still}, mirroring the mPFC results from our group.

\section{Learning to Time}
    Most of the work in Timing has focused on well-trained subjects \cite{}, what limits the community ability to select models. We introduced in section \ref{sec:models} models that have explicit mechanisms for learning.
    
% \section{Timing is young}
% We want to explore the relationship between the area
% Timing is a young field - compared to learning.
%  we should engage with both
