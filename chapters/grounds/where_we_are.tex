\begin{savequote}[75mm]
In every discourse, whether of the mind conversing with its own thoughts, or of the individual in his intercourse with others, there is an assumed or expressed limit within which the subjects of its operation are confined. The most unfettered discourse is that in which the words we use are understood in the widest possible application, and for them the limits of discourse are co-extensive with those of the universe itself.
\qauthor{George Boole}
\end{savequote}

\chapter{Where we are}
\label{chap:where}

In this chapter we trace a smaller circle around our contributions, 


\section{Our vantage point}
% Now we can assess learning of timing in rats in the brain activity.
Electrode recording techniques have important limitations: They record only a small subset of neurons, and the specific neurons being recorded may change with passing days. While the low density of recordings can be bypassed with homogeneity assumptions, it makes harder to test hypothesis on recorded data, specially when the identity of neurons is transitional. One way to tackle the limitations is to compare recordings within a day, but this is only interesting if behavior changes fast, in the course of a single session.
    \subsection{An unique task}
    % It was not possible. Problems. DRRD task
    Time learning is a slow process \cite{paton2018neural}. To study learning of temporal representations in the cortex, our group devised changes to the DRRD protocol that enabled learning to occur in a matter of some hours \cite{}. This new protocol enables to compare the activity of the same neurons between moments in the session. 
    
    \subsection{Multivariate analysis techniques}
    To assess multidimensional representations, we should use multivariate techniques. Machine learning provides general multivariate techniques, capable of answering questions with little assumptions with respect to the data. Instead of searching for number of time cells or ramping neurons, we can ask about representation in a more general sense, and measure its presence in the recorded neural activity. 


% \section{Potential contributions}
As introduced before, we will split our theoretical contributions into their corresponding level of analysis whenever possible. 
    \subsection{Computational}
    % Computational
        % Taxonomy of timing
        Multiple learning mechanisms seem to play their roles in animal behavior. 
    \subsection{Algorithmic}
    % Algorithmic
        % processes
        % learning mechanisms
    \subsection{Implementational}
    % Implementational
        % brain regions
        % neural correlates
        
\section{Objectives}

\section{Hypothesis}