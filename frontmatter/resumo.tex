Nosso grupo desenvolveu um protocolo de treinamdento de Reforço Diferencial de Duração de Resposta que permite aos ratos aprender em uma única sessão. Consequentemente, nós gravamos a atividade neural em duas regiões: O Córtex Pré Frontal medial e o Estriado. Nós buscamos iluminar o processo de desenvolvimento de representações nessas areas. Para isso, usamos aprendizado de máquina para prever o tempo corrido desde o início de uma resposta sustentada, usando como input a atividade neuronal instantânea. Nós então medimos a performance do algoritmo de predição, e associamos performances maiores com uma melhor representação de tempo. Nós encontramos que os algoritmos conseguem decodificar o tempo a partir da atividade de neurônios no Córtex Pré Frontal medial mesmo no início da sessão de treinamento. No entanto, a performance do classificador reduz tanto durante a primeira sessão quanto em um segundo dia de treinamento. Na direção oposta, a representação do tempo no Estriado, também mensurada pela performance de decodificação, melhora com o treinamento. Tal evidência indica que o Córtex Pré Frontal medial é necessário somente para o aprendizado e progressivamente desengaja da tarefa, enquanto há um aumento progressivo do envolvimento do Estriado ao longo do aprendizado. Nossos resultados são consistentes com a hipótese de que o treinamento leva os ratos por um proceso de habituação e pode ajudar a elucidar o papel do Córtex Pré Frontal medial e do Estriado na representação interna do tempo. Além disso, nossa metodologia de aprendizado de máquina oferece uma maneira simples de mensurar representações neurais multivariadas, e pode ser utilizada em outros contextos não relacionados com o estudo de timing.

\textbf{Palavras-chave}: aprendizagem temporal; DRRD; aprendizagem de máquina; córtex pré-frontal; estriado; intervalos de tempo.