%!TEX root = ../dissertation.tex
% the abstract

Our group has developed a training protocol for the Differential Reinforcement of Response Duration task that allows rats to learn in a single session. Hence, we recorded the neural activity in two regions: the medial Pre Frontal Cortex (mPFC) and the Striatum. We aimed to shed some light on the process of learning to time by studying how time representations develop in these areas. For this purpose, we used machine learning to predict the time elapsed since each sustained response had started, using as input the instantaneous neuronal spiking activity. We then measured the performance of the algorithm and associated higher performance with better time representation. We found that the algorithm can decode the time from the spiking activity from neurons in the mPFC a the beginning of the training session. However, such performance decreases both during the first training session and on the second day of training. In the opposite direction, the representation of time in the Striatum, also measured in terms of decoding performance, enhances with training. Such evidence indicates that the mPFC is necessary only for learning and progressively disengages the task, while there is a progressive involvement of the Striatum throughout learning. Our findings are consistent with the hypothesis that training drives rats into a process of habituation and may help to elucidate the role of the mPFC and the Striatum in the internal representation of time. Moreover, our machine learning methodology is a simple way to assess neural representations and can be used in other contexts unrelated to timing.

\textbf{Palavras-chave}: temporal learning; DRRD; machine-learning; prefrontal cortex; striatum; interval timing.